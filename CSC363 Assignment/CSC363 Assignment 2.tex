\documentclass{csc_assignment}
\usepackage[utf8]{inputenc}
\usepackage[letterpaper, portrait, margin=1in]{geometry}
\usepackage{calc}  % arithmetic in length parameters
\usepackage{enumitem}  % more control over list formatting
\usepackage{fancyhdr}  % simpler headers and footers
\usepackage{lastpage}  % for last page number
\usepackage{relsize}  % easier font size changes
\usepackage[normalem]{ulem}  % smarter underlining
\usepackage{url}  % verb-like typesetting of URLs
\usepackage{xfrac}  % nicer looking simple fractions for text and math
\usepackage{amsmath}
\usepackage{amssymb}
\usepackage{tikz}
\usepackage{algorithm}
\usepackage{algorithmic}

% ----------------------------------------------------------------
% TODO: Enter the assignment number, your name, and your student number below
% ----------------------------------------------------------------
\AssignmentName{2}
\QuestionCount{5}
\StudentName{Akhil Gupta}
\StudentNumber{1000357071}

% ----------------------------------------------------------------
\begin{document}


\Acknowledgements{
% ----------------------------------------------------------------
% TODO: Write your acknowledgements below.
% ----------------------------------------------------------------

%Replace this text with a list acknowledging any outside help you have received in completing this assignment (outside help means: the textbook, the notes on the course website, the teaching assistants, and the instructor). 
%For example, if you have discussed or collaborated with another student on a question, state the question number and the name of the student. If you haven't received any help write:

"I declare that I have not used any outside help (excluding the textbook, the notes on the course website, the teaching assistants, and the instructor) in completing this assignment."

% ----------------------------------------------------------------
% Acknowledgements ends
% ----------------------------------------------------------------
}
\begin{description}

\newpage
\item[Q1.]
% ----------------------------------------------------------------
% TODO: Write your answer to the question below. 
% ----------------------------------------------------------------

In this question we give an alternative proof that \emph{nondeterministic} Turing Machines compute exactly the languages in \emph{SD}.
  (For the definition of a nondeterministic Turing Machine, check the Sipser book or the notes from Tutorial 2).
  Fix an alphabet $\Sigma$.
  Give a {\bf direct} proof that for any language $L$, there is a nondeterministic Turing Machine $M$ such that $L = \mathcal{L}(M)$ if and only if there is a computable relation $R \subseteq \Sigma^* \times \Sigma^*$ such that \[ x \in L \Leftrightarrow \exists y \in \Sigma^*: R(x, y).\] \\
  Answer: (adapted from Professor Robert's Lecture 4 notes) \\ Let $\Sigma = \{0,1\}$. First we prove the ``if'' direction. Let $L$ be a semi-decidable language, and let $M_0$ be a
Non-Deterministic Turing Machine such that $\mathcal{L}(M_0)$ = $L$. Theorem 2 from Tutorial 2 Notes states that: \\For any language $L$, $L \in SD$ if and only if there is an Non-Deterministic Turing Machine that computes $L$  \\ For any $(x, y) \in \Sigma^*$ x $\Sigma^*$, define $R$ by \\ $R(x, y)$ $\Leftrightarrow$ $y$ encodes all possible computations of $M_0$ on $x$.\\[2pt]
Next we show that $x \in L$ if and only if $(x, y) \in R$.\\
If $x \in L$, then $M_0$ accepts $x$, and so there must be some sequence of configurations that leads to the string $y$ encoding an accepting computation of $M_0$ on $x$. Thus if $x \in L$ then there is a $y$ such that $R(x, y)$ holds. Conversely, if there is a $y$ that encodes an accepting computation of $M_0$ on $x$, then there must be some sequence of configurations that leads $M_0$ to accept $x$. So if $R(x, y)$ holds then $x \in L$.\\[2pt]
The algorithm for $R$ operates as follows:\\[1pt]
1. On input $x \in \Sigma^*$.\\
2. Nondeterministically construct arbitary strings $y_1, y_2, y_3, \ldots$ $y \in \Sigma^{*}$ and encode all possible configurations of the Nondeterministic Turing Machine $M_0$ i.e create a computational tree where the children nodes of a node are $M_{0}$'s next configurations\\
3. Repeat the following:\\
(a) Simulate $M_0$ on $x$ for one step, and check if the configuration of $M_0$ on $x$ is encoded at some branch of $y$'s computational tree. If not, reject.\\
(b) If this is the last configuration encoded in $y$, check that $M_0$ has actually halted and
that there is a sequence of configurations that leads to an accept state i.e. $M_{0}$ accepts $x$. If so, accept. Otherwise, reject.\\
Clearly the algorithm above halts on all of its inputs i.e. all the branches of $y$'s computational tree halt, as it rejects as soon as the simulation of $M_0$ on $x$ does not find a match in $y$'s computational tree. Moreover, the algorithm only accepts if and only if there is a sequence of accept state configurations that leads to  string $y$ that encodes an accepting computation that $M_0$ accepts $x$, and thus if and only if $(x, y) \in R$.\\[2pt]
Now we prove the ``only if'' direction in the statement of the theorem. Suppose that such a decidable relation $R$ exists, and we use the relation $R$ to construct a Non-Deterministic Turing Machine $M_0$ which recognizes the language $L$. Let $M$ be the Non-Determinsitic Turing Machine deciding $R$. \\ The machine $M_0$ will run the following algorithm:\\
1. On input $x \in \Sigma^*$.\\
2. For each $y \in \Sigma^*$ in y's computational tree (from the algorithm above).\\
(a) Simulate $M$ on the input $(x, y)$. If there exists some sequence of accept states that leads to $x$, then $M$ accepts, Otherwise continue.\\ Since $R$ is decidable the nondeterminsitic machine $M$ halts on all inputs i.e. all branches halt on all inputs, so each simulation step in the for loop will halt. The algorithm above accepts $x \in \Sigma^*$ if and only if there is a $y \in \Sigma^*$ in y's computational tree such that $R(x, y)$ holds; by assumption, this is equivalent to $x \in L$. If no such $y$ exists, then the algorithm will never halt. Thus $\mathcal{L}(M_0)$ = $\mathcal{L}$ and so $L \in SD$.

% ----------------------------------------------------------------
% Answer ends
% ----------------------------------------------------------------

\item[Q2.a]
% ----------------------------------------------------------------
% TODO: Write your answer to the question below. 
% ----------------------------------------------------------------

  Let $\Sigma = \{0,1\}$.
  For each of the following languages $L \subseteq \Sigma^*$, classify $L$ with respect to \emph{D}, \emph{SD}, \emph{coSD}.
  That is, for each language $L$ and for each class $\mathsf{C} \in \{\emph{D}, \emph{SD}, \emph{coSD}\}$, prove that $L$ is in $\mathsf{C}$ or that $L$ is not in $\mathsf{C}$.
  {\bf You may not use Rice's Theorem.}
  \begin{enumerate}
  \item $L_1 = \{(\langle{M}\rangle, \langle{i}\rangle) | M \text{ is a TM, } i \in \mathbb{N}, \text{ M accepts all strings of length }i\}$ \\
  Answer: We need to show that for all $i \in \mathbb{N}$, $M$ accepts all strings of length $i$. According to Professor Robert's A1Q4 Solutions, if we are given $\langle i \rangle$, we can compute $f(i)$ ($f$ is a computable function). So if we can compute $f(i)$, we can also do a little more work and get $i$. Hence, in order to solve this, we will reduce the Halting Problem to $L_{1}$. The Halting Problem is defined as $$\{(\langle M \rangle, w) | M \text{ is a TM and } M \text{ halts on } w\}$$ We know that the Halting Problem $\in SD$, therefore, we will prove that $L_{1} \in SD$. \\ Let $(\langle M, w \rangle)$ be any pair such that $M$ is a TM, and we show how to construct a Turing Machine $M'_{(M,w)}$ such that $(\langle M, w \rangle) \in \overline{\text{HALT}}$ $\Leftrightarrow$ $(\langle M'_{(M,w)}\rangle) \in$ $L_{1}$. \\ Consider the following algorithm $M'_{(M,w)}$ which is computable from the pair $(\langle M \rangle, w)$.\\ Algorithm for $M'_{(M,w)}$: \\ 1. On input $x \in \Sigma^{*}$, skip the input and write $w$. \\ 2. Simulate $M$ on $w$ for at most $i$ steps ($i$ can be computed from $\langle i \rangle$). \\ 3. Accept if and only if $|w|$ = $i$. \\ If $(\langle M, w \rangle) \in$ HALT, then $M$ halt on $w$ $\Leftrightarrow$ the algorithm $M'_{(M,w)}$ accepts all input including all strings of length $i$ $\Leftrightarrow$ $(\langle M'_{(M,w)}\rangle) \in$ $L_{1}$. If $(\langle M, w \rangle) \notin$ HALT, then $M$ does not halt on $w$ $\Leftrightarrow$ the algorithm $M'_{(M,w)}$ does not accept any input particularly ones of length $i$ $\Leftrightarrow$ $(\langle M'_{(M,w)}\rangle) \notin$ $L_{1}$. It follows that HALT $\leq_{m}$ $L_{1}$, and since HALT $\in SD$, this shows that $L_{1} \in SD$. Since $L_{1} \in SD$, it immediately shows that $L_{1} \notin coSD$ and $\notin D$ since $D = SD \cap coSD$.
  
  
% ----------------------------------------------------------------
% Answer ends
% ----------------------------------------------------------------

\item[Q2.b]

  \item $L_2 = \{(\langle{M}\rangle, x) | M \text{ is a TM and } M \text{ halts on } x \text{ with } 11111 \text{ written on the tape\}}$ \\
  Answer: We solve this problem by reducing the Halting Problem to $L_{2}$. The Halting Problem is defined as $$\{(\langle M \rangle, w) | M \text{ is a TM and } M \text{ halts on } w\}$$ We know that the Halting Problem $\in SD, \notin D$, therefore, we will prove that $L_{2}$ $\in SD, \notin D$.\\ Let $(\langle M \rangle, w)$ be any pair such that $M$ is a TM, and we show how to construct a Turing Machine $M'_{(M,w)}$ such that $(\langle M \rangle, w) \in$ HALT $\Leftrightarrow$ $(M'_{(M,w)}) \in$ $L_{2}$. \\ Consider the following algorithm $M'_{(M,w)}$ which is computable from the pair $(\langle M \rangle, w)$.\\ Algorithm for $M'_{(M,w)}$: \\ 1. On input $x \in \Sigma^{*}$. \\ 2. Skip the input and write $w$ on the tape. Then simulate $M$ on $w$. \\ 3. Halt and accept if any of the simulations have $11111$ written on its tape and blanks everywhere else, else reject. \\ If $(\langle M \rangle, w) \in$ HALT, then $M$ halts on $w$ so the algorithm $M'_{(M,w)}$ halts on input $x$ with $11111$ written on the tape.\\ This means $M'_{(M,w)} \in L_{2}$. On the other hand, if $M'_{(M,w)} \in L_{2}$, then by definition of $M'_{(M,w)}$, it is easy to see that $M'_{(M,w)}$ halts on any input if and only if $M$ halts on $w$. Thus $(\langle M \rangle, w) \in$ HALT. It follows that HALT $\leq_{m}$ $L_{2}$, and since HALT $\in SD$ and $\notin D$, this shows that $L_{2} \in SD$ and $\notin D$. Since $L_{2} \in SD$, it immediately shows that $L_{2} \notin coSD$ as well since $D = SD \cap coSD$ (from Professor Robert's lecture notes).
  
% ----------------------------------------------------------------
% Answer ends
% ----------------------------------------------------------------
\item[Q2.c]

  \item $L_3 = \{\langle{M}\rangle | M \text{ is a TM and }\exists i \in \mathbb{N}: \text{M accepts all strings of length }i\}$ \\
  Answer: We need to show that there exists some $i \in \mathbb{N}$ for which $M$ accepts all strings of length $i$. In order to solve this, we will reduce the $\overline{\text{Halting Problem}}$ to $L_{3}$. The $\overline{\text{Halting Problem}}$ is defined as $$\{(\langle M \rangle, w) | M \text{ is a TM and } M \text{ does not halt on } w\}$$ We know that the $\overline{\text{Halting Problem}}$ $\notin SD$, therefore, we will prove that $L_{3}$ $\notin SD$. \\ Let $(\langle M, w \rangle)$ be any pair such that $M$ is a TM, and we show how to construct a Turing Machine $M'_{(M,w)}$ such that $(\langle M, w \rangle) \in \overline{\text{HALT}}$ $\Leftrightarrow$ $(\langle M'_{(M,w)}\rangle) \in$ $L_{3}$. \\ Consider the following algorithm $M'_{(M,w)}$ which is computable from the pair $(\langle M \rangle, w)$.\\ Algorithm for $M'_{(M,w)}$: \\ 1. On input $x \in \Sigma^{*}$. \\ 2. Skip $x$ and write $w$. Then simulate $M$ on $w$ for all inputs for at most $|x|$ steps. \\ 3. Reject if $M$ halts on $w$ within $|x|$ steps and accept otherwise. \\ If $(\langle M, w \rangle) \in \overline{\text{HALT}}$, then $M$ does not halt on $w$ $\Leftrightarrow$ the algorithm $M'_{(M,w)}$ accepts all input including all strings of length $i$ $\Leftrightarrow$ $(\langle M'_{(M,w)}\rangle) \in$ $L_{3}$. On the other hand, if $M'_{(M,w)} \in L_{3}$, then by definition of $M'_{(M,w)}$, it is easy to see that $M'_{(M,w)}$ halts and accepts all input of length $i$ if and only if $M$ halts on $w$ $\Leftrightarrow$ $(\langle M \rangle, w) \in$ $\overline{\text{HALT}}$. It follows that $\overline{\text{HALT}}$ $\leq_{m}$ $L_{3}$, and since $\overline{\text{HALT}}\notin SD$, this shows that $L_{3} \notin SD$. Since $L_{3} \notin SD$, it immediately shows that $L_{3} \in coSD$ and $\notin D$ since $D = SD \cap coSD$. \\[50pt] part d) is on the next page $\rightarrow$
  
% ----------------------------------------------------------------
% Answer ends
% ----------------------------------------------------------------
\newpage
\item[Q2.d]

  \item $L_4 = \{\langle{M}\rangle | M \text{ is a TM and there is an } x \in \Sigma^* \text{ beginning with } 0 \text{ such that } M \text{ does not accept }x\}$ \\
  Answer:  We have to show that there exists some $x \in \Sigma^{*}$ beginning with 0 such that $M$ does not accept $x$. In order to solve this, we will reduce the $\overline{\text{Halting Problem}}$ to $L_{4}$. The $\overline{\text{Halting Problem}}$ is defined as $$\{(\langle M \rangle, w) | M \text{ is a TM and } M \text{ does not halt on } w\}$$ We know that the $\overline{\text{Halting Problem}}$ $\notin SD$, therefore, we will prove that $L_{4}$ $\notin SD$. \\ Let $(\langle M, w \rangle)$ be any pair such that $M$ is a TM, and we show how to construct a Turing Machine $M'_{(M,w)}$ such that $(\langle M, w \rangle) \in \overline{\text{HALT}}$ $\Leftrightarrow$ $(\langle M'_{(M,w)}\rangle) \in$ $L_{4}$. \\ Consider the following algorithm $M'_{(M,w)}$ which is computable from the pair $(\langle M \rangle, w)$.\\ Algorithm for $M'_{(M,w)}$: \\ 1. On input $x \in \Sigma^{*}$. \\ 2. Simulate $M$ on $w$. \\ 3. If $M$ halts on $w$, $M'_{(M,w)}$ accepts.  \\ If $(\langle M, w \rangle) \in \overline{\text{HALT}}$, then $M$ does not halt on $w$ $\Leftrightarrow$ the algorithm $M'_{(M,w)}$ does not accept any input particularly ones beginning with $0$  $\Leftrightarrow$ $(\langle M'_{(M,w)}\rangle) \in$ $L_{4}$. If $(\langle M, w \rangle) \notin \overline{\text{HALT}}$, then $M$ halts on $w$ $\Leftrightarrow$ the algorithm $M'_{(M,w)}$ accepts any input particularly ones beginning with $0$  $\Leftrightarrow$ $(\langle M'_{(M,w)}\rangle) \notin$ $L_{4}$. It follows that $\overline{\text{HALT}}$ $\leq_{m}$ $L_{4}$, and since $\overline{\text{HALT}}\notin SD$, this shows that $L_{4} \notin SD$.\\[3pt]                        
The complement of $L_{4}$ = $\{\langle{M}\rangle | M \text{ does not encode a TM or for all } x \in \Sigma^* \text{ beginning with } 0, M \text{ accepts }x\}$. \\ We now prove that $\overline{L_{4}} \notin SD$. Assume that $\langle M \rangle$ is a well-formed encoding of a turing machine then according to the description of the language, it accepts all strings in $\Sigma^{*}$ beginning with $0$ $\Leftrightarrow$ $\overline{L_{4}}$ contains infinitely many strings i.e. $\mathcal{L}(\overline{L_4}) = \text{infinite}$. So now we have to prove $\overline{L_4} = \{\langle{M}\rangle | M \text{ is a TM and } \mathcal{L}(\overline{L_4}) = \text{infinite}\}$ $\notin SD$. From Professor Robert's tutorial 5 notes, we proved that INF $\notin SD$ via a reduction from $A_{TM}$*.\\ (INF = $\{\langle M \rangle | M \text{ is a TM that accepts infinitely many strings} \}$). \\ So now we have that $L_{4} \notin SD$ and $\overline{L_4} \notin SD$, it immediately shows that $L_{4} \notin coSD$ and subsequently $\notin D$ since $D = SD \cap coSD$. Therefore, $L_{4} \notin SD, \notin coSD$, and therfore $\notin C$.
  \end{enumerate}
  
% ----------------------------------------------------------------
% Answer ends
% ----------------------------------------------------------------


\end{description}
\end{document}
  
  



